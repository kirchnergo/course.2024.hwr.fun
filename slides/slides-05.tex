% Created 2024-05-03 Fri 11:42
% Intended LaTeX compiler: pdflatex
\documentclass[t]{beamer}
\usepackage[utf8]{inputenc}
\usepackage[T1]{fontenc}
\usepackage{graphicx}
\usepackage{longtable}
\usepackage{wrapfig}
\usepackage{rotating}
\usepackage[normalem]{ulem}
\usepackage{amsmath}
\usepackage{amssymb}
\usepackage{capt-of}
\usepackage{hyperref}
\mode<beamer>{\usetheme{Amsterdam}}
\mode<beamer>{\usecolortheme{rose}}
\usepackage{fontspec}
\usepackage{polyglossia}
\setmainlanguage[babelshorthands=true]{german}
\usepackage{hyperref}
\usepackage{color}
\usepackage{xcolor}
\usepackage[misc]{ifsym}
\definecolor{darkblue}{rgb}{0,0,.5}
\definecolor{darkgreen}{rgb}{0,.5,0}
\definecolor{islamicgreen}{rgb}{0.0, 0.56, 0.0}
\definecolor{darkred}{rgb}{0.5,0,0}
\definecolor{mintedbg}{rgb}{0.95,0.95,0.95}
\definecolor{arsenic}{rgb}{0.23, 0.27, 0.29}
\definecolor{prussianblue}{rgb}{0.0, 0.19, 0.33}
\definecolor{coolblack}{rgb}{0.0, 0.18, 0.39}
\hypersetup{colorlinks=true, breaklinks=true, anchorcolor=blue,linkcolor=white, citecolor=islamicgreen, filecolor=darkred,  urlcolor=darkblue}
\usepackage{booktabs}
\usepackage{pgf}
\usepackage{minted}
\RequirePackage{fancyvrb}
\DefineVerbatimEnvironment{verbatim}{Verbatim}{fontsize=\scriptsize}
\usetheme{default}
\author{Göran Kirchner\thanks{e\_kirchnerg@doz.hwr-berlin.de}}
\date{2024-05-03}
\title{Funktionale Programmierung in F\# (5)}
\subtitle{Parser Combinators}
\hypersetup{
 pdfauthor={Göran Kirchner},
 pdftitle={Funktionale Programmierung in F\# (5)},
 pdfkeywords={},
 pdfsubject={},
 pdfcreator={Emacs 29.1 (Org mode 9.6.6)}, 
 pdflang={English}}
\begin{document}

\maketitle

\section{Ziel }
\label{sec:orgbbdf8c2}
\begin{frame}[label={sec:org7e7496f}]{Programm}
\begin{itemize}
\item Programmieraufgabe
\item Test
\item Parser (Kombinatoren)
\end{itemize}
\end{frame}

\section{Programmieraufgabe }
\label{sec:org9a68bc1}
\begin{frame}[label={sec:orgb982682},fragile]{Poker}
 \begin{minted}[bgcolor=mintedbg,frame=none,framesep=0pt,mathescape=true,fontsize=\scriptsize,breaklines=true,linenos=false,numbersep=5pt,gobble=0]{fsharp}
type Rank =
     | Two | Three | Four | Five | Six | Seven | Eight | Nine | Ten 
     | Jack | Queen | King | Ace
type HandCategory =
     | HighCard of Rank * Rank * Rank * Rank * Rank
     | OnePair of Rank * Rank * Rank * Rank
     | TwoPair of Rank * Rank * Rank
     | ThreeKind of Rank * Rank
     | Straight of Rank
     | Flush of Rank
     | FullHouse of Rank * Rank
     | FourKind of Rank * Rank
     | StraightFlush of Rank
     | RoyalFlush
\end{minted}
\end{frame}

\begin{frame}[label={sec:org9273532},fragile]{Zusammenfassung}
 \begin{itemize}
\item nutze \href{https://exercism.io}{exercism.io}!
\item Vermeide \texttt{mutable}!!
\item nur wichtiges verdient einen Namen
\item Vertraue der \alert{Pipe} (\texttt{>{}>{}}, \texttt{|>}, \ldots{})!!
\item If-Then-Else mit Boolean ist unnötig
\item Parametrisiere!
\item If-Then-Else vermeiden \ldots{} besser \texttt{match}!
\item Be lazy! (vermeide \texttt{for}-loops)
\item \href{https://fsharpforfunandprofit.com/troubleshooting-fsharp/}{Troubleshooting F\#}
\item \href{https://docs.microsoft.com/de-de/dotnet/fsharp/style-guide/}{F\#-Styleguide}
\end{itemize}
\end{frame}

\section{Test }
\label{sec:orgc3075cb}
\begin{frame}[label={sec:org1a1c00b}]{Test}
\begin{itemize}
\item 60 Minuten
\end{itemize}

\(\leadsto\) \href{../src/test/test.md}{Test}
\end{frame}

\section{Parser }
\label{sec:org5594484}
\begin{frame}[label={sec:org750c248},fragile]{Parser 1 (hard-coded character)}
 \begin{minted}[bgcolor=mintedbg,frame=none,framesep=0pt,mathescape=true,fontsize=\scriptsize,breaklines=true,linenos=false,numbersep=5pt,gobble=0]{fsharp}
open System
let A_Parser str =
    if String.IsNullOrEmpty(str) then
        (false,"")
    else if str.[0] = 'A' then
        let remaining = str.[1..]
        (true,remaining)
    else
        (false,str)
let inputABC = "ABCD"
let inputZBC = "ZBCD"
let test11 = A_Parser inputABC
let test12 = A_Parser inputZBC
\end{minted}

\begin{verbatim}
val A_Parser: str: string -> bool * string
val inputABC: string = "ABCD"
val inputZBC: string = "ZBCD"
val test11: bool * string = (true, "BCD")
val test12: bool * string = (false, "ZBCD")
\end{verbatim}
\end{frame}

\begin{frame}[label={sec:orgbe2c99f},fragile]{Parser 2 (match a specified character)}
 \begin{minted}[bgcolor=mintedbg,frame=none,framesep=0pt,mathescape=true,fontsize=\scriptsize,breaklines=true,linenos=false,numbersep=5pt,gobble=0]{fsharp}
let pchar (charToMatch,str) =
    if String.IsNullOrEmpty(str) then
        let msg = "No more input"
        (msg,"")
    else 
        let first = str.[0] 
        if first = charToMatch then
            let remaining = str.[1..]
            let msg = sprintf "Found %c" charToMatch
            (msg,remaining)
        else
            let msg = sprintf "Expecting '%c'. Got '%c'" charToMatch first
            (msg,str)
\end{minted}

\begin{verbatim}
val pchar: charToMatch: char * str: string -> string * string
\end{verbatim}
\end{frame}

\begin{frame}[label={sec:org360c6cc},fragile]{Parser 2 (2)}
 \begin{minted}[bgcolor=mintedbg,frame=none,framesep=0pt,mathescape=true,fontsize=\scriptsize,breaklines=true,linenos=false,numbersep=5pt,gobble=0]{fsharp}
let inputABC = "ABCD"
let inputZBC = "ZBCD"
let test21 = pchar('A',inputABC) 
let test22 = pchar('A',inputZBC)
\end{minted}

\begin{verbatim}
val inputABC: string = "ABCD"
val inputZBC: string = "ZBCD"
val test21: string * string = ("Found A", "BCD")
val test22: string * string = ("Expecting 'A'. Got 'Z'", "ZBCD")
\end{verbatim}
\end{frame}

\begin{frame}[label={sec:org7cb0886},fragile]{Parser 3 (return a Result)}
 \begin{minted}[bgcolor=mintedbg,frame=none,framesep=0pt,mathescape=true,fontsize=\scriptsize,breaklines=true,linenos=false,numbersep=5pt,gobble=0]{fsharp}
let pchar (charToMatch, s) =
    if String.IsNullOrEmpty(s) then
        Error "No more input"
    else
        let first = s.[0]
        if first = charToMatch then
            let remaining = s.[1..]
            Ok (charToMatch, remaining)
        else
            let msg = sprintf "Expecting '%c'. Got '%c'" charToMatch first
            Error msg
\end{minted}

\begin{verbatim}
val pchar: charToMatch: char * s: string -> Result<(char * string),string>
\end{verbatim}
\end{frame}

\begin{frame}[label={sec:orgeacafaf},fragile]{Parser 3 (2)}
 \begin{minted}[bgcolor=mintedbg,frame=none,framesep=0pt,mathescape=true,fontsize=\scriptsize,breaklines=true,linenos=false,numbersep=5pt,gobble=0]{fsharp}
let test31 = pchar('A',inputABC) 
let test32 = pchar('A',inputZBC) 
let test33 = pchar('Z',inputZBC)
\end{minted}

\begin{verbatim}
val test31: Result<(char * string),string> = Ok ('A', "BCD")
val test32: Result<(char * string),string> = Error "Expecting 'A'. Got 'Z'"
val test33: Result<(char * string),string> = Ok ('Z', "BCD")
\end{verbatim}
\end{frame}

\begin{frame}[label={sec:org7b98d91},fragile]{Parser 4 (use currying)}
 \begin{minted}[bgcolor=mintedbg,frame=none,framesep=0pt,mathescape=true,fontsize=\scriptsize,breaklines=true,linenos=false,numbersep=5pt,gobble=0]{fsharp}
let pchar charToMatch str = 
    if String.IsNullOrEmpty(str) then
        Error "No more input"
    else
        let first = str.[0] 
        if first = charToMatch then
            let remaining = str.[1..]
            Ok (charToMatch,remaining)
        else
            let msg = sprintf "Expecting '%c'. Got '%c'" charToMatch first
            Error msg
\end{minted}

\begin{verbatim}
val pchar: charToMatch: char -> str: string -> Result<(char * string),string>
\end{verbatim}
\end{frame}

\begin{frame}[label={sec:org7f066a4},fragile]{Parser 4 (2)}
 \begin{minted}[bgcolor=mintedbg,frame=none,framesep=0pt,mathescape=true,fontsize=\scriptsize,breaklines=true,linenos=false,numbersep=5pt,gobble=0]{fsharp}
let parseA = pchar 'A'
let inputABC = "ABC"
let inputZBC = "ZBC"
let test41 = parseA inputABC
let test42 = parseA inputZBC
let parseZ = pchar 'Z' 
let test43 = parseZ inputZBC
\end{minted}

\begin{verbatim}
val parseA: (string -> Result<(char * string),string>)
val inputABC: string = "ABC"
val inputZBC: string = "ZBC"
val test41: Result<(char * string),string> = Ok ('A', "BC")
val test42: Result<(char * string),string> = Error "Expecting 'A'. Got 'Z'"
val parseZ: (string -> Result<(char * string),string>)
val test43: Result<(char * string),string> = Ok ('Z', "BC")
\end{verbatim}
\end{frame}

\begin{frame}[label={sec:org5b296fc},fragile]{Parser 5 (type to wrap the parser function)}
 \begin{minted}[bgcolor=mintedbg,frame=none,framesep=0pt,mathescape=true,fontsize=\scriptsize,breaklines=true,linenos=false,numbersep=5pt,gobble=0]{fsharp}
type Parser<'T> =
    | Parser of (string -> Result<'T , string>)
let pchar charToMatch = 
    let innerFn str =
        if String.IsNullOrEmpty(str) then
            Error "No more input"
        else
            let first = str.[0] 
            if first = charToMatch then
                let remaining = str.[1..]
                Ok (charToMatch, remaining)
            else
                let msg = sprintf "Expecting '%c'. Got '%c'" charToMatch first
                Error msg
    Parser innerFn
\end{minted}

\begin{verbatim}
type Parser<'T> = | Parser of (string -> Result<'T,string>)
val pchar: charToMatch: char -> Parser<char * string>
\end{verbatim}
\end{frame}

\begin{frame}[label={sec:orgea80157},fragile]{Parser 5 (2)}
 \begin{minted}[bgcolor=mintedbg,frame=none,framesep=0pt,mathescape=true,fontsize=\scriptsize,breaklines=true,linenos=false,numbersep=5pt,gobble=0]{fsharp}
let parseA = pchar 'A'
let inputABC = "ABC"
parseA inputABC
\end{minted}

\begin{verbatim}

  parseA inputABC;;
  ^^^^^^

error FS0003: This value is not a function and cannot be applied.
\end{verbatim}
\end{frame}

\begin{frame}[label={sec:org499de12},fragile]{Parser 5 (3)}
 \begin{minted}[bgcolor=mintedbg,frame=none,framesep=0pt,mathescape=true,fontsize=\scriptsize,breaklines=true,linenos=false,numbersep=5pt,gobble=0]{fsharp}
let run parser input = 
    let (Parser innerFn) = parser 
    innerFn input
let parseA = pchar 'A' 
let inputABC = "ABC"
let test1 = run parseA inputABC
let inputZBC = "ZBC"
let test2 = run parseA inputZBC
\end{minted}

\begin{verbatim}
val run: parser: Parser<'a> -> input: string -> Result<'a,string>
val parseA: Parser<char * string> = Parser <fun:pchar@238-14>
val inputABC: string = "ABC"
val test1: Result<(char * string),string> = Ok ('A', "BC")
val inputZBC: string = "ZBC"
val test2: Result<(char * string),string> = Error "Expecting 'A'. Got 'Z'"
\end{verbatim}
\end{frame}

\section{Parser Kombinatoren }
\label{sec:orgcd35902}
\begin{frame}[label={sec:orgb49b178}]{Understanding Parser Combinators}
\(\leadsto\) \href{./5 Understanding parser combinators.pdf}{Understanding parser combinators} (Scott Wlashin)

\null\hfill--Scott Wlashin: \href{https://fsharpforfunandprofit.com/parser}{F\# for Fun and Profit}
\end{frame}

\begin{frame}[label={sec:org7a79f22}]{FParsec Tutorial}
\begin{itemize}
\item \href{http://www.quanttec.com/fparsec/tutorial.html\#}{FParsec Tutorial}
\item \href{http://www.quanttec.com/fparsec/users-guide/}{User’s Guide}
\item \href{http://www.quanttec.com/fparsec/about/fparsec-vs-alternatives.html}{FParsec vs alternatives}
\end{itemize}
\end{frame}

\begin{frame}[label={sec:org9c3497d},fragile]{Using FParsec (1)}
 \begin{minted}[bgcolor=mintedbg,frame=none,framesep=0pt,mathescape=true,fontsize=\scriptsize,breaklines=true,linenos=false,numbersep=5pt,gobble=0]{fsharp}
#r "../src/5/02-fparsec/lib/FParsecCS.dll";; 
#r "../src/5/02-fparsec/lib/FParsec.dll";;
open FParsec
let test p str =
    match run p str with
    | Success(result, _, _)   -> printfn "Success: %A" result
    | Failure(errorMsg, _, _) -> printfn "Failure: %s" errorMsg;;
test pfloat "1.25"
test pfloat "1.25E 2"
\end{minted}

\begin{verbatim}
test pfloat "1.25"
test pfloat "1.25E 2";;
Success: 1.25                           
Failure: Error in Ln: 1 Col: 6
1.25E 2
     ^
Expecting: decimal digit

val it: unit = ()
\end{verbatim}
\end{frame}

\begin{frame}[label={sec:orgfce0df4},fragile]{Using FParsec (2)}
 \begin{minted}[bgcolor=mintedbg,frame=none,framesep=0pt,mathescape=true,fontsize=\scriptsize,breaklines=true,linenos=false,numbersep=5pt,gobble=0]{fsharp}
let str s = pstring s
let floatBetweenBrackets:Parser<float, unit>  = str "[" >>. pfloat .>> str "]";;

test floatBetweenBrackets "[1.0]"
test floatBetweenBrackets "[]"
test floatBetweenBrackets "[1.0]"
\end{minted}

\begin{verbatim}
test floatBetweenBrackets "[1.0]"
test floatBetweenBrackets "[]"
test floatBetweenBrackets "[1.0]";;
Success: 1.0                            
Failure: Error in Ln: 1 Col: 2
[]
 ^
Expecting: floating-point number

Success: 1.0                            
val it: unit = ()
\end{verbatim}
\end{frame}

\begin{frame}[label={sec:org8356b84},fragile]{Using FParsec (3)}
 \begin{minted}[bgcolor=mintedbg,frame=none,framesep=0pt,mathescape=true,fontsize=\scriptsize,breaklines=true,linenos=false,numbersep=5pt,gobble=0]{fsharp}
let betweenStrings s1 s2 p = str s1 >>. p .>> str s2;;
let floatBetweenBrackets_:Parser<float, unit> = pfloat |> betweenStrings "[" "]";;
let floatBetweenDoubleBrackets_:Parser<float, unit> = pfloat |> betweenStrings "[[" "]]";;
test floatBetweenBrackets_ "[1.0]"
test floatBetweenDoubleBrackets_ "[[1.0]]"
let between_ pBegin pEnd p  = pBegin >>. p .>> pEnd;;
let betweenStrings_ s1 s2 p = p |> between_ (str s1) (str s2);;
test (many floatBetweenBrackets) ""
test (many floatBetweenBrackets) "[1.0]"
test (many floatBetweenBrackets) "[2][3][4]"
test (many floatBetweenBrackets) "[1][2.0E]"
\end{minted}

\begin{verbatim}
Success: []
Success: [1.0]
Success: [2.0; 3.0; 4.0]
Failure: Error in Ln: 1 Col: 9
[1][2.0E]
        ^
Expecting: decimal digit

val it: unit = ()
\end{verbatim}
\end{frame}

\section{Ende }
\label{sec:org1f0db14}
\begin{frame}[label={sec:org8b71561}]{Zusammenfassung (Kurs)}
\begin{itemize}
\item Wichtige Werkzeuge (git, dotnet, code)
\item Elementare Syntax
\item Funktionen, Pattern Matching, Discriminated Unions (DU)
\item Tuple, Record, List, Array, Seq
\item funktionale Operationen auf Listen (Tail-Rekursion)
\item funktionaler Umgang mit fehlenden Daten (Option)
\item funktionaler Umgang mit Fehlern (Result)
\item funktionales Design (statt Patterns: Funktionen \& Verkettung)
\item funktionales Refactoring
\item funktionales Domain Modeling (DDD)
\item eigenschaftsbasiertes Testen (Property Based Testing) (cool!!)
\item funktionale Parser (Kombinatoren) (noch cooler!!)
\end{itemize}
\(\leadsto\) \alert{\alert{Was ist Funktionale Programmierung?}}
\end{frame}

\begin{frame}[label={sec:orgb0480b3}]{Links}
\begin{itemize}
\item \href{https://fsharp.org/}{fsharp.org}
\item \href{https://docs.microsoft.com/de-de/dotnet/fsharp/}{docs.microsoft.com/../dotnet/fsharp}
\item \href{https://sergeytihon.com/}{F\# weekly}
\item \href{https://fsharpforfunandprofit.com/}{fsharpforfunandprofit.com}
\item \href{https://github.com/fsprojects/awesome-fsharp}{github.com/../awesome-fsharp}
\end{itemize}
\end{frame}

\begin{frame}[label={sec:orgfaba227}]{Ende}
\begin{itemize}
\item Wie geht es weiter?
\item \href{https://exercism.io}{Exercism!}
\item Buchtipps
\begin{itemize}
\item \href{https://pragprog.com/book/swdddf/domain-modeling-made-functional}{Domain Modeling Made Functional} (F\#)
\item \href{https://www.apress.com/gp/book/9781484239995}{Stylish F\#} (F\#)
\item \href{https://www.cambridge.org/core/books/pearls-of-functional-algorithm-design/B0CF0AC5A205AF9491298684113B088F\#}{Perls of Functional Algorithm Design} (Haskell)
\item \href{https://www.cs.ox.ac.uk/publications/books/functional/}{Thinking Functional with Haskell} (Haskell)
\item \href{http://www.paulgraham.com/onlisp.html}{On Lisp} (LISP)
\item \href{http://www.iqool.de/FPMP.html}{Funktionale Programmierung und Metaprogrammierung} (LISP)
\item \href{https://github.com/norvig/paip-lisp}{Paradigms of Artificial Intelligence Programming} (LISP)
\item \href{https://adv-r.hadley.nz/}{Advanced R} (R)
\end{itemize}
\item Sprachen: \href{https://www.r-project.org/}{R}, \href{https://www.haskell.org/}{Haskell}, \href{https://clojure.org/}{Clojure}, \href{https://lisp-lang.org/}{Common Lisp}, \href{https://elixir-lang.org/}{Elixir}, \href{https://code.kx.com/q/}{q}

\item \alert{\alert{Have FUN!}}
\end{itemize}
\end{frame}
\end{document}